This thesis proposes \textit{SLDC}, a generic framework for object detection and classification in mutli-gigapixel images. 
It provides implementers with a concise way of formulating their algorithm by declaring only problem dependent-components: segmentation procedures and classification models. Behind the scenes, the framework takes care of problem-independent concerns. For instance, in order to avoid loading the full image into memory, it splits this image in tiles which are processed independently. Parallelism is also encapsulated by the framework which applies it to accelerate tiles processing. Are also provided: a powerful and customizable logging system informing the user about errors and overall progress, a way of executing several workflows one after another on a same image and robustness so that errors of which the impact is negligible does not stop the whole program. The framework is available on GitHub as a Python library. 

The framework was then applied to a real-world problem, thyroid nodule malignancy diagnosis, in order to assess its performances. Especially, a workflow developed for this problem in a previous master thesis was analysed, improved and re-implemented using the framework. 

The results are promising: the effective execution time of the workflow was less than 10 minutes on a 8 gigapixels image (executed on 32 processes). This time is mostly due to design choice linked to the implementation and the framework only induces a negligible overhead. Some improvements can be done both at the framework and workflow levels. Especially, some other operations of the former could be parallelized and the current parallelization could be optimized.

As far as the thyroid case is concerned, the developed workflow does not provide a production-ready solution yet because it sometimes fails at detecting objects of interest and produces an important number of false positives. However, the analysis provided in this thesis already points out elements which needs to be improved (segmentation procedures, classification models,...) providing a baseline for any further development on this case.
